\documentclass[modern]{aastex62}

% lsstdoc documentation: https://lsst-texmf.lsst.io/lsstdoc.html
\input{meta}

% Package imports go here.

\usepackage{tcolorbox}

% Local commands go here.

\newcommand{\note}[1]{
   \begin{tcolorbox}[colback=red!5!white, colframe=red!75!black]
      #1
   \end{tcolorbox}
}



\newcommand{\docRef}{PSTN-022}
\newcommand{\docUpstreamLocation}{\url{https://github.com/lsst-pst/pstn-022}}


\begin{document}
\input{authors}
\date{\today}
\title{LSST Science Platform}
\hypersetup{pdftitle={\@title}, pdfauthor={\@author}, pdfkeywords={\@keywords}}


\begin{abstract}
 
As the Commissioning Execution Plan (LSE-390) says, "The project team shall
deliver all reports documenting the as-built hardware and software including:
drawings, source code, modifications, compliance exceptions, and recommendations
for improvement." As a first step towards the delivery of documents that will describe the system at the
end of construction, we are assembling teams for producing of the order 40 papers
that eventually will be submitted to relevant professional journals. The immediate goal is to accomplish
all the writing that can be done without data analysis before the data
taking begins, and the team becomes much more busy and stressed.

This document provides the template for these papers.
\end{abstract}



\section{Introduction}

\section{High-level User-Facing description - better title to be devised}

\note{This section will present a suitably modified version of the high-level picture from the LSE-319 LSP Vision document.  This includes introducing the concept of the Aspects and the users' ability to work cross-Aspect.  It also includes a high-level view of the performance requirements (e.g., number of users supported, total CPU and storage available, volume of queries supported).}

Key figures and tables: ``cartoon''-level 3-Aspect diagram; list of IVOA standards supported; major performance requirements and design points

\section{Functional Architecture}

\note{This section will describe the functional architecture of the LSP, enumerating the Aspects and their major components, the connections between them, and their interfaces to other LSST systems, to users, and to the external world.}

\subsection{Major Components}

Key figures and tables: Aspect-level data flow diagram showing connections of the LSP to other components of the LSSST and to each other, and to users; ``internal block diagram'' showing a breakdown of each Aspect into its top

\subsection{Data Flows and Interfaces}

\note{This includes setting out the key externally-visible interfaces.}

Key figures and tables: ``internal block diagram'' showing a breakdown of each Aspect into its top-level components and showing interfaces at that level, as well as support components such as A&A

\section{Common Deployment Architecture}

\note{This section describes the common deployment architecture for the LSP, e.g., Kubernetes, containerization, virtualized networking, etc., including setting out any common details that will be needed to be referenced in the Aspect-specific sections that follow.  This section also defines the LSP deployment instances, internal as well as cloud, as appropriate to the final design.  It includes a high-level description of the way that A&A is applied in the LSP, as well as a discussion of the cyber-security considerations relevant to the LSP.}

Key figures and tables: diagram supporting the explanation of key Kubernetes concepts such as namespaces, pods, and containers; diagram showing the relationship of a representative LSP instance's Kubernetes cluster to other LDF resources; diagram and/or table supporting the presentation of the various LSP instances

\subsection{LSP Instances}

\note{Defines the standing LSP instances and their operational roles.}

\subsection{Deployment at the LDF}

\note{Defines details associated with running the LSP instances that are located at the LDF.}

\subsection{User management and cyber-security: overview}

\note{Describes the security and A&A architecture sufficiently to support references to it in the Aspect-specific sections.  A fuller description follows below.}

\section{TBC: The Data Model of the Archival Data Products}

\note{This text --- whether ultimately in this paper or another one --- will describe the data model of the released data products that are accessible to users: how they are laid out in databases and file/object stores, and how metadata about the data products are stored and made available to the Aspects.  Depending on the level of detail elsewhere this section might be just a summary or might be a lengthy presentation.  I am assuming, however, that the upstream Science Data Model Standardization that leads to the released Data Model will be described in a Pipelines paper.}

\section{The Notebook Aspect}

\subsection{JupyterLab and the notebook model}

\subsection{LSST's extensions to JupyterLab and the integration with the Science Pipelines stack}

\subsection{Usage illustrations}

\section{The Portal Aspect}

\subsection{Firefly}

\subsection{LSP-specific Portal components}

\subsection{Usage illustrations}

\section{The API Aspect}

\subsection{Catalog services}

\subsection{Image metadata services}

\subsection{Image services}

\subsection{Python interfaces to the API Aspect}

\subsection{Usage illustrations}

\section{Authentication, Authorization, and Identity Management; Cyber-security}

\note{Describes the federated-identity, single-sign-on A&A model used in the LSP in some technical detail.  A HIGHLY technical paper on this might still be published separately in a suitable technical journal.}

Key figures and tables: block diagram of the proxy architecture used to support A&A; interaction/sequence diagram showing the transaction and token flow involved in login and access to interactive web-based services; same for programmatic services

\section{LSP Computing Resources}

\note{This section provides additional details on the computing and storage resources at the LDF and/or in the cloud that support the LSP.}

Key figures and tables: Enumeration of 

\subsection{Resource Management considerations}

\note{Discusses user quotas and the resource-request mechanism(s).}

\subsection{Batch Computing}

\note{Depending on boundaries with other papers, may either just briefly summarize, or present in some detail, the next-to-the-data computing environment.}

\section{Operational Experience and Testing}

\note{This section presents operational experience with the LSP during I&T and Commissioning, as well as formal testing and informal testing and user-experience-gathering from precursor and Stack-Club-ish activities.  It includes lessons learned and notes on how this experience refined the design being presented.}

\section{Appendix: Key Use Cases}

\note{This section will include a selection of key use cases that drive either the overall functional architecture of the LSP, the major design choices for a single Aspect, or the performance requirements on the Aspects.}

\section{Appendix: IVOA and Community Standards Supported}

\note{This section will contain a list, and a short context-setting paragraph for each entry, of the use and role of each of the supported IVOA and community (e.g., CAOM2) standards supported by the LSP.}



\appendix
% Remove this when you strart your paper

{\bf Initial paper list added here for reference.}

``Editor'' is a responsible team leader but not necessarily the person who will do most of
the required work, or who will eventually become the first author. Both issues will be
handled by individual teams.

\begin{verbatim}

domain: Telescope & Site
editor: Jeff Barr
title: Overview of the LSST Telescope

domain: Telescope & Site
editor: Sandrine Thomas
title: Performance of the LSST Telescope

domain: Telescope & Site
editor: Lynne Jones
title: The LSST Scheduler Overview and Performance

domain: Telescope & Site
editor: Bo Xin
title: Performance of the LSST Active Optics System

domain: Telescope & Site
editor: Tiago Ribeiro
title: LSST Observing System Software Architecture

domain: Camera
editor: Justin Wolfe
title: LSST Camera Optics

domain: Camera
editor: Chris Stubbs
title: LSST Camera Rafts

domain: Camera
editor: Steve Ritz
title: LSST Camera Cryostat

domain: Camera
editor: Ralph Schindler
title: LSST Camera Refrigeration

domain: Camera
editor: Steve Ritz
title: LSST Camera Body and Mechanisms

domain: Camera
editor: Mark Huffer and Tony Johnson
title: LSST Camera Control System and DAQ

domain: Camera
editor: Tim Bond and Aaron Rodman
title: LSST Camera Integration and Tests

domain: Data Management
editor: Leanne Guy
title: Overview of LSST Data Management

domain: Data Management
editor: Michelle Butler
title: LSST Data Facility

domain: Data Management
editor: Tim Jenness
title: LSST Data Management Software System

domain: Data Management
editor: Jim Bosch
title: LSST Data Release Processing

domain: Data Management
editor: Eric Bellm
title: LSST Prompt Data Products

domain: Data Management
editor: Gregory Dubois-Felsmann
title: LSST Science Platform

domain: Data Management
editor: Simon Krughoff
title: LSST Data Management Quality Assurance and Reliability Engineering

domain: Data Management
editor: Leanne Guy (with likely delegation to new DM V&V Scientist)
title: LSST Data Management System Verification and Validation

domain: Data Management
editor: Mario Juric
title: LSST Moving Object Processing

domain: Data Management
editor: Robert Lupton
title: LSST Calibration Strategy and Pipelines

domain: Calibration
editor: Patrick Ingraham
title:  Performance of the LSST Calibration Systems

domain: Calibration
editor: Patrick Ingraham
title: Atmospheric Properties with the LSST Auxiliary Telescope

domain: EPO
editor: Amanda Bauer
title: Overview of LSST Education and Public Outreach

domain: EPO
editor: Ardis Herrold
title: LSST Formal Education Program

domain: EPO
editor: Amanda Bauer
title: LSST EPO: The User Feedback

domain: Commissioning
editor: Chuck Claver
title: LSST Observatory System Operations Readiness Report

domain: Commissioning
editor: Bo Xin
title: Performance of Delivered LSST System

domain: Commissioning
editor: Chuck Claver
title: Active Optics Performance with LSST Commissiong Camera

domain: Commissioning
editor: Chuck Claver
title: LSST Active Optics Performance with the LSST Science Camera

domain: Commissioning
editor: Brian Stalder
title: Integration, Test and Commissioning Results from LSST Commissiong Camera

domain: Commissioning
editor: Kevin Reil
title: LSST Camera Instrumental Signature Characterization, Calibration and Removal

domain: Commissioning
editor: Patrick Hascal
title: Installation and Performance of the LSST Camera Refrigeration System

domain: Commissioning
editor: Andy Connolly
title: Science Validation of LSST Alert Processing

domain: Commissioning
editor: Keith Bechtol
title: Science Validation of LSST Data Release Processing

domain: Commissioning
editor: Michael Reuter
title: Tracking of LSST System Performance with Continuous Integration Methods

domain: Commissioning
editor: Chuck Claver
title: The LSST Science Platform as a Commissioning Tool

domain: Commissioning
editor: Chuck Claver
title: Commissioning Science Data Quality Analysis Tools, Methods and Procedures

domain: Commissioning
editor: Lynne Jones
title: Performance Verification of the LSST Survey Scheduler


\end{verbatim}

% Include all the relevant bib files.
% https://lsst-texmf.lsst.io/lsstdoc.html#bibliographies
\section{References} \label{sec:bib}
\bibliographystyle{yahapj}
\bibliography{local,lsst,lsst-dm,refs_ads,refs,books}

% Make sure lsst-texmf/bin/generateAcronyms.py is in your path
\section{Acronyms} \label{sec:acronyms}
\input{acronyms.tex}

\end{document}
