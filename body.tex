\section{Introduction}

\section{High-level User-Facing description - better title to be devised}

\note{This section will present a suitably modified version of the high-level picture from the LSE-319 LSP Vision document.  This includes introducing the concept of the Aspects and the users' ability to work cross-Aspect.  It also includes a high-level view of the performance requirements (e.g., number of users supported, total CPU and storage available, volume of queries supported).}

Key figures and tables: ``cartoon''-level 3-Aspect diagram; list of IVOA standards supported; major performance requirements and design points

\section{Functional Architecture}

\note{This section will describe the functional architecture of the LSP, enumerating the Aspects and their major components, the connections between them, and their interfaces to other LSST systems, to users, and to the external world.}

\subsection{Major Components}

Key figures and tables: Aspect-level data flow diagram showing connections of the LSP to other components of the LSSST and to each other, and to users; ``internal block diagram'' showing a breakdown of each Aspect into its top

\subsection{Data Flows and Interfaces}

\note{This includes setting out the key externally-visible interfaces.}

Key figures and tables: ``internal block diagram'' showing a breakdown of each Aspect into its top-level components and showing interfaces at that level, as well as support components such as A&A

\section{Common Deployment Architecture}

\note{This section describes the common deployment architecture for the LSP, e.g., Kubernetes, containerization, virtualized networking, etc., including setting out any common details that will be needed to be referenced in the Aspect-specific sections that follow.  This section also defines the LSP deployment instances, internal as well as cloud, as appropriate to the final design.  It includes a high-level description of the way that A&A is applied in the LSP, as well as a discussion of the cyber-security considerations relevant to the LSP.}

Key figures and tables: diagram supporting the explanation of key Kubernetes concepts such as namespaces, pods, and containers; diagram showing the relationship of a representative LSP instance's Kubernetes cluster to other LDF resources; diagram and/or table supporting the presentation of the various LSP instances

\subsection{LSP Instances}

\note{Defines the standing LSP instances and their operational roles.}

\subsection{Deployment at the LDF}

\note{Defines details associated with running the LSP instances that are located at the LDF.}

\subsection{User management and cyber-security: overview}

\note{Describes the security and A&A architecture sufficiently to support references to it in the Aspect-specific sections.  A fuller description follows below.}

\section{TBC: The Data Model of the Archival Data Products}

\note{This text --- whether ultimately in this paper or another one --- will describe the data model of the released data products that are accessible to users: how they are laid out in databases and file/object stores, and how metadata about the data products are stored and made available to the Aspects.  Depending on the level of detail elsewhere this section might be just a summary or might be a lengthy presentation.  I am assuming, however, that the upstream Science Data Model Standardization that leads to the released Data Model will be described in a Pipelines paper.}

\section{The Notebook Aspect}

\subsection{JupyterLab and the notebook model}

\subsection{LSST's extensions to JupyterLab and the integration with the Science Pipelines stack}

\subsection{Usage illustrations}

\section{The Portal Aspect}

\subsection{Firefly}

\subsection{LSP-specific Portal components}

\subsection{Usage illustrations}

\section{The API Aspect}

\subsection{Catalog services}

\subsection{Image metadata services}

\subsection{Image services}

\subsection{Python interfaces to the API Aspect}

\subsection{Usage illustrations}

\section{Authentication, Authorization, and Identity Management; Cyber-security}

\note{Describes the federated-identity, single-sign-on A&A model used in the LSP in some technical detail.  A HIGHLY technical paper on this might still be published separately in a suitable technical journal.}

Key figures and tables: block diagram of the proxy architecture used to support A&A; interaction/sequence diagram showing the transaction and token flow involved in login and access to interactive web-based services; same for programmatic services

\section{LSP Computing Resources}

\note{This section provides additional details on the computing and storage resources at the LDF and/or in the cloud that support the LSP.}

Key figures and tables: Enumeration of 

\subsection{Resource Management considerations}

\note{Discusses user quotas and the resource-request mechanism(s).}

\subsection{Batch Computing}

\note{Depending on boundaries with other papers, may either just briefly summarize, or present in some detail, the next-to-the-data computing environment.}

\section{Operational Experience and Testing}

\note{This section presents operational experience with the LSP during I&T and Commissioning, as well as formal testing and informal testing and user-experience-gathering from precursor and Stack-Club-ish activities.  It includes lessons learned and notes on how this experience refined the design being presented.}

\section{Appendix: Key Use Cases}

\note{This section will include a selection of key use cases that drive either the overall functional architecture of the LSP, the major design choices for a single Aspect, or the performance requirements on the Aspects.}

\section{Appendix: IVOA and Community Standards Supported}

\note{This section will contain a list, and a short context-setting paragraph for each entry, of the use and role of each of the supported IVOA and community (e.g., CAOM2) standards supported by the LSP.}

