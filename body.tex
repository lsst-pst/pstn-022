\section{Introduction}

likely author: Gregory

\section{High-level User-Facing description - better title to be devised}

likely author: Gregory

\note{This section will present a suitably modified version of the high-level picture from the LSE-319 LSP Vision document.  This includes introducing the concept of the Aspects and the users' ability to work cross-Aspect.  It also includes a high-level view of the performance requirements (e.g., number of users supported, total CPU and storage available, volume of queries supported).}

Key figures and tables: ``cartoon''-level 3-Aspect diagram; list of IVOA standards supported; major performance requirements and design points

\section{Functional Architecture}

likely author: Frossie

\note{This section will describe the functional architecture of the LSP, enumerating the Aspects and their major components, the connections between them, and their interfaces to other LSST systems, to users, and to the external world.}

\subsection{Major Components}

Key figures and tables: Aspect-level data flow diagram showing connections of the LSP to other components of the LSSST and to each other, and to users; ``internal block diagram'' showing a breakdown of each Aspect into its top

\subsection{Data Flows and Interfaces}

\note{This includes setting out the key externally-visible interfaces.}

Key figures and tables: ``internal block diagram'' showing a breakdown of each Aspect into its top-level components and showing interfaces at that level, as well as support components such as A&A

\section{Common Deployment Architecture}

likely author: SQuaRE team member

\note{This section describes the common deployment architecture for the LSP, e.g., Kubernetes, containerization, virtualized networking, etc., including setting out any common details that will be needed to be referenced in the Aspect-specific sections that follow.  This section also defines the LSP deployment instances, internal as well as cloud, as appropriate to the final design.  It includes a high-level description of the way that A&A is applied in the LSP, as well as a discussion of the cyber-security considerations relevant to the LSP.}

Key figures and tables: diagram supporting the explanation of key Kubernetes concepts such as namespaces, pods, and containers; diagram showing the relationship of a representative LSP instance's Kubernetes cluster to other LDF resources; diagram and/or table supporting the presentation of the various LSP instances

\subsection{LSP Instances}

\note{Defines the standing LSP instances and their operational roles.}

\subsection{Deployment at the LDF}

\note{Defines details associated with running the LSP instances that are located at the LDF.}

\subsection{User management and cyber-security: overview}

\note{Describes the security and A&A architecture sufficiently to support references to it in the Aspect-specific sections.  A fuller description follows below.}

\section{TBC: The Data Model of the Archival Data Products}

likely author: Colin or Gregory

\note{This text --- whether ultimately in this paper or another one --- will describe the data model of the released data products that are accessible to users: how they are laid out in databases and file/object stores, and how metadata about the data products are stored and made available to the Aspects.  Depending on the level of detail elsewhere this section might be just a summary or might be a lengthy presentation.  I am assuming, however, that the upstream Science Data Model Standardization that leads to the released Data Model will be described in a Pipelines paper.}

Key figures and tables: graphical illustration of the major catalog data products, including visual display of the key cross-table links (foreign key relationships), as well as relationships between these tables and the file-oriented data collections; table with headline numbers on current and expected data volumes

\section{The Notebook Aspect}

likely authors: SQuaRE team member, Simon

\note{This section briefly describes JupyterLab and the notebook model, just for context and with references, and then goes on to describe how this was integrated with the LSST development and release environment, and the Science Pipelines stack.  }

Key figures and tables: JupyterHub-JupyterLab relationship and workspaces; at least one additional TBD figure

\subsection{JupyterLab and the notebook model}

\subsection{LSST's JupyterLab environment and the integration with the Science Pipelines stack}

\note{This subsection describes LSST's extensions to JupyterLab and defines the user environment experienced by users who connect to it.}

\subsection{Usage illustrations}

\note{This subsection presents screenshots and narratives illustrating particularly interesting elements of how the Notebook Aspect can be used.}

\section{The Portal Aspect}

likely author: Gregory

\note{This section describes the Portal Aspect and the Firefly software on which it depends.  The depth of discussion of Firefly will depend on the then-year availability of a suitable existing Firefly publication that could be referenced.  It describes how Firefly is interfaced to LSST data services and how it runs within the LSP deployment architecture.  It presents any Portal-specific issues associated with A&A, such as the forwarding of tokens to the User File Workspace and TAP services.}

Key figures and tables: diagram illustrating the client-server visualization architecture used in Firefly; diagram illustrating the connection of the Portal Aspect Firefly server(s) to other LSST components

\subsection{Firefly}

\subsection{LSP-specific Portal components}

\subsection{Usage illustrations}

\note{This subsection presents screenshots and narratives illustrating particularly interesting elements of how the Portal Aspect can be used.}

\section{The API Aspect}

likely authors: Colin, Christine, Gregory

\note{This section describes the API services that make the released data products available.  It relies on the existence of a separate paper which discusses the databases in detail, as well as (TBC) issues such as the internal implementation of DAX services such as image cutouts.  It discusses how the services interoperate, including elements such as their use of DataLink and other metadata for connecting requests to different services.  It also discusses the Python interfaces available, from the community as well as from LSST, for interaction with the API service; this covers in particular any special feature of Python access in the Notebook Aspect, including a comparison of purely Butler-based interactions with API-service-based interactions.}

Key figures and tables: functional block diagrams and/or sequence diagrams that illustrate certain key multi-step accesses to the service, such as an image metadata query followed by an image data query, or an Object catalog query followed by a query for time series data for selected Objects

\subsection{Catalog services}

This primarily refers to TAP, as well as to any lightweight wrapper services, such as a DataLink-style timeseries query service.  It needs to cover ther services available for interaction with the User Database Workspace, i.e., for user table upload and federation.

\subsection{Image metadata services}

This primarily refers to ObsTAP and SIAv2 services, as well as to any lightweight support services, e.g., those related to DataLink.

\subsection{Image services}

This refers to services for actual image data, including data in their primary units of storage (i.e., CCD-level and patch-level images), as well as for cutouts (via SODA), and for recreated-on-demand image data.  It also covers the HiPS and MOC services provided by the project.

\subsection{File workspace services}

This refers to services supporting the User File Workspace, such as WebDAV and VOSpace.  It describes the relationship between the API service view of this file storage and the mounted-filesystem view provided in the Notebook Aspect.

Key figures and tables: illustration of the relationships between these services and mounted filesystems in the Notebook Aspect

\subsection{Python interfaces to the API Aspect}

This section covers the availability of Python wrappers for the API services, including PyVO and any other community-standard libraries that work with the API services.  It also covers Project-specific Python interfaces that may be provided.  Finally, it compares and contrasts Python access to the API services with Butler data access in the Notebook Aspect, to a level that assists users in understanding which to use.

\subsection{Usage illustrations}

\note{This subsection presents screenshots and narratives illustrating particularly interesting elements of how the API Aspect can be used.}

\section{Authentication, Authorization, and Identity Management; Cyber-security}

likely authors: Russ and Brian

\note{Describes the federated-identity, single-sign-on A&A model used in the LSP in some technical detail.  A HIGHLY technical paper on this might still be published separately in a suitable technical journal.}

Key figures and tables: block diagram of the proxy architecture used to support A&A; interaction/sequence diagram showing the transaction and token flow involved in login and access to interactive web-based services; same for programmatic services

\section{LSP Computing Resources}

likely author: LDF team member

\note{This section provides additional details on the computing and storage resources at the LDF and/or in the cloud that support the LSP.}

Key figures and tables: Enumeration of physical resources available to the user-visible LSP instances

\subsection{Resource Management considerations}

\note{Discusses user quotas and the resource-request mechanism(s).}

\subsection{Batch Computing}

\note{Depending on boundaries with other papers, may either just briefly summarize, or present in some detail, the next-to-the-data computing environment.}

\section{Operational Experience and Testing}

likely author: Frossie

\note{This section presents operational experience with the LSP during I&T and Commissioning, as well as formal testing and informal testing and user-experience-gathering from precursor and Stack-Club-ish activities.  It includes lessons learned and notes on how this experience refined the design being presented.}

\section{Appendix: Key Use Cases}

likely authors: Gregory, Simon, Colin

\note{This section will include a selection of key use cases that drive either the overall functional architecture of the LSP, the major design choices for a single Aspect, or the performance requirements on the Aspects.}

\section{Appendix: IVOA and Community Standards Supported}

\note{This section will contain a list, and a short context-setting paragraph for each entry, of the use and role of each of the supported IVOA and community (e.g., CAOM2) standards supported by the LSP.}

